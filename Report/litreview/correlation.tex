% CORRELATION
\subsection{Correlation}
\label{subsection:corr}
The application of a linear filter $h(u,v)$ to an image $f(i,j)$ may be described as follows

\begin{equation} \label{eq:1}
g(i,j) = \sum_{u=-k}^{k}\sum_{v = -l}^{l}f(i+u,j+v)h(u,v)
\end{equation}

$g(i,j)$ is the output image. Performing correlation with a filter may be notated more concisely by the correlation operator.

\[g = f \otimes h\]

Correlation measures the similarity between two signals. Both digital images and linear filters are two dimensional signals. Performing correlation between them will yield an output image where the highest values correspond to where the image and filter were most similar \cite{optimalKernel}. This useful because if you wish to emphasise a feature in an image it can be done by correlating it with a filter that describes that feature. For example, if you wished to exagerate vertical and horizontal lines you could use Sobel filters as in \ref{fig:sobel_filters}. 

% SOBEL MASKS
\begin{figure}[H]
  \begin{subfigure}[b]{0.49\textwidth}
    \[
    \begin{bmatrix*}[l]
     -1 & -1 & -1 \\
      \phantom{-}2 & \phantom{-}2 & \phantom{-}2 \\
      -1 & -1 & -1 
    \end{bmatrix*}
    \]
    \caption{Horizontal Sobel Filter Mask}
    \label{rfidtest_xaxis}
\end{subfigure}
\begin{subfigure}[b]{0.49\textwidth}
  \[ 
    \begin{bmatrix}
      -1 & 2 & -1 \\
      -1 & 2 & -1 \\
      -1 & 2 & -1
    \end{bmatrix}
    \]
    \caption{Vertical Sobel Filter Mask}  
\end{subfigure}
    \caption{Sobel Filters}
    \label{fig:sobel_filters}
\end{figure}

The use of negative weightings means that values next to an edge are diminished and the positively weighted line sections (the 2s) strengthen line features. Notice in figure \ref{fig:sobel_apply} how the lines have a high values (white) and regions that aren't lines are low valued (black).

% SOBEL FILTER APPLICATION
\begin{figure}[H]
  \centering
  \begin{subfigure}[b]{0.3\textwidth}
      \includegraphics[width=\textwidth]{im_color}
      \caption{Image by Simone Hutsch}
  \end{subfigure}
  \begin{subfigure}[b]{0.3\textwidth}
      \includegraphics[width=\textwidth]{gv}
      \caption{Vertical Sobel Filter}
      \label{fig:vert}
  \end{subfigure}
  \begin{subfigure}[b]{0.3\textwidth}
      \includegraphics[width=\textwidth]{gh}
      \caption{Horizontal Sobel Filter}
      \label{fig:hoz}
  \end{subfigure}
  \caption{Application of Sobel filters to exagerate lines.}
  \label{fig:sobel_apply}
\end{figure}

Correlation is \emph{shift invariant}, which means that it does the same thing no matter where in an image it is applied. To satisfy this property correlation may be superpositioned 

\[a(f_1 + f_2) = af_1 + af_2\]

and abides by the shift invariance principle

\[g(i,j)=f(i+k,j+l) \Leftrightarrow\ (h\circ g)(i,j)=(h\circ f)(i+k,j+l)\]

Correlation has the side effect of flipping both horizontally and vertically the location of output points relative to location the center point (\emph{reference point}) in the original image which may be undesirable.

% CORRELATION EXAMPLE  %
\begin{figure}[H] 
  \centering
  \begin{tabular}{ccccc}
      \begin{tabular}{|c|c|c|c|c|}
      \hline
      0 & 0 & 0 & 0 & 0 \\[1pt]
      \hline
      0 & 0 & 0 & 0 & 0 \\[1pt]
      \hline
      0 & 0 & 1 & 0 & 0 \\[1pt]
      \hline
      0 & 0 & 0 & 0 & 0 \\[1pt]
      \hline
      0 & 0 & 0 & 0 & 0 \\[1pt]
      \hline
      \end{tabular}%
    & $\otimes$ &
    \begin{tabular}{|c|c|c|}
      \hline
      a & b & c \\
      \hline
      d & e & f \\
      \hline
      g & h & i \\
      \hline 
    \end{tabular}
    & $=$ &
    \begin{tabular}{|c|c|c|c|c|}
      \hline
      0 & 0 & 0 & 0 & 0 \\[1pt]
      \hline
      0 & \textbf{i} & \textbf{h} & \textbf{g} & 0 \\[1pt]
      \hline
      0 & \textbf{f} & \textbf{e} & \textbf{d} & 0 \\[1pt]
      \hline
      0 & \textbf{c} & \textbf{b} & \textbf{a} & 0 \\[1pt]
      \hline
      0 & 0 & 0 & 0 & 0 \\[1pt]
      \hline
    \end{tabular} \\
    $F(x,y)$ & & $H(u,v)$& & $G(x,y)$ \\
  \end{tabular}
  \caption{Correlation of a filter and an image.}
  \label{fig:correlation}
\end{figure}