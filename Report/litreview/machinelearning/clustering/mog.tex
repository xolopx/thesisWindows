\subsubsection{Mixtures of Gaussians}
 \label{subsection:mog}

Mixtures of Gaussians (MoG), or the Gaussian Mixture Model (GMM), is a method of clustering (see section \ref{subsection:clustering}) that is based not on distance \ref{subsubsection:kmeans} but distribution. This method of clustering was popularized by Duda and Hart in their text \emph{Pattern Classification and Scene Analysis} \cite{mog_seminal}. This method is more effective than distance based methods because it considers the covariance of the data. Note that for K-Means clustering the distributions were spherical, but if the true clusters were more elliptical then K-Means would fail to cluster them properly. Figure \ref{fig:cluster_shapes} compares what these cluster shapes might look like and the arrows indicate the location, the arrows indicate data points that are ambiguous.

\begin{figure}[h]
	\centering
	\includegraphics[width = 0.9\textwidth]{litreview/computervision/mog/mog_vs_kmeans.png}
	\captionsetup{format = hang}
	\caption{Comparison of K-Means cluster shape to GMM cluster shape. Img: Hongning Wang, University of Virginia }
	\label{fig:cluster_shapes}
\end{figure}
	
The basis of the Gaussian Mixture Model is of course the Gaussian Distribution function \ref{section:gaussian}. It is useful for probabilistic clustering because, as the Central Limit Theorem States, the distribution of the average value of the subsets of any data set will approximately converge to a Normal distribution as the number of terms in the subsets increases \cite{patterns_machine_learning}. The Mixture of Gaussian for the random variable $\boldsymbol{x}$ is achieved by super-positioning the distributions of random variable for each cluster. Equation \ref{eq:mog} describes the super-positioning of $K$ distributions for random variable $\boldsymbol{x}$, the resulting curve models the distribution of the random variable. 

\begin{equation}
p(\boldsymbol{x}) = \sum^{K}_{k = 1}\pi_k \mathcal{N}(\boldsymbol{x}|\boldsymbol{\mu_k}, \boldsymbol{\Sigma_k})
\label{eq:mog}
\end{equation}

The variable $\pi_k$ is known as the \emph{mixing coefficient} of a Gaussian distribution which controls the weighting of a particular distribution in the mixture. The sum of the mixing ratios is 1 because the complete mixture distribution represent the probability of a sample belonging to any one cluster and it must belong to one, hence

\[\sum_{k=1}^K\pi_k = 1\]

To help to understand how equation \ref{eq:mog} came about consider the random variable $\boldsymbol{z}$ of dimension K which is a binary random variable, meaning it can only have value 1 or 0. Only one element of $\boldsymbol{z}$ can be equal to 1 and all others must be 0, thus

\[\sum_{k=1}^Kz_k = 1\]

Hence there are K possible combinations of the vector $\boldsymbol{z}$. The marginal distribution over $\boldsymbol{z}$ is defined in terms of the mixing coefficients, 

\[p(z_k = 1) = \pi_k\]

That is to say the probability of $z_k$ = 1 is equal to the mixing coefficient of Gaussian distribution k. This can also be written

\[p(\boldsymbol{z}) = \prod^K_{k=1}\pi^{z_k}_k\]

The significance of $\boldsymbol{z}$ represents the assignment of a sample to a cluster which is why only one of its elements can be 1. To observe the Gaussian distribution of $\boldsymbol{z}$ in cluster $K$ we set $z_k = 1$ as described by the condition distribution,

\[p(\bm{x}|\boldsymbol{z}) = \mathcal{N}(\boldsymbol{x}|\boldsymbol{\mu_k}, \boldsymbol{\Sigma_k}) \]

Finally we can define the marginal distribution of $\boldsymbol{x}$, formerly described by \ref{eq:mog}, using the marginal distribution $p(\boldsymbol{z})$ and the conditional distribution $p(\bm{x}|\bm{z})$ as described by equation \ref{eq:mog_derive}. Figure \ref{fig:mog_compare} shows the super-positioning of individual Gaussian distributions to form a mixture of Gaussian for a one-dimensional dataset.

\begin{align}
	p(\bm{x}) 	&= p(\bm{z})p(\bm{x}|\bm{z})\\
				&= \sum^{K}_{k = 1}\pi_k \mathcal{N}(\boldsymbol{x}|\boldsymbol{\mu_k}, \boldsymbol{\Sigma_k})
\label{eq:mog_derive}
\end{align}


\begin{figure}[htbp]
    \centering
     \begin{subfigure}[b]{0.45\textwidth}
        \includegraphics[width=\textwidth]{litreview/machinelearning/clustering/mog/individual_gauss.png}
	\captionsetup{format = hang}
        \caption{Three individual cluster distributions.}
        \label{fig:mog_singles}
    \end{subfigure} 
    \begin{subfigure}[b]{0.45\textwidth}
        \includegraphics[width=\textwidth]{litreview/machinelearning/clustering/mog/combined_mog.png}	
	\captionsetup{format = hang}
        \caption{Super-positioned Gaussian distributions}
        \label{fig:mog_combined}
    \end{subfigure}
    \captionsetup{format = hang}
    \caption{Visualization of a Mixture of Gaussians for 1D dataset.}
    \label{fig:mog_compare}
\end{figure}

\textbf{Expectation Maximization}

Initially the Gaussian's that represent each cluster in the mixture model have random parameters, like the centroids for K-Means clustering (\ref{subsubsection:kmeans}) but the objective of the mixture model is to represent the distribution for a dataset, $\bm{x}$. Expectation Maximization \cite{dempster_EM}\cite{emalgo} is the algorithm that determines how to update each Gaussian's parameters. It seeks to maximise the likelihood of any one sample, $x_i$, belonging to a Gaussian which is achieved by maximizing the log likelihood function. The log likelihood is just the log of the likelihood function of the mixture model (\ref{eq:mog}) and is described by equation \ref{eq:log_likelihood}. Expectation maximization gives each sample a soft-assignment to a cluster and then changes each cluster distribution's parameters, $\bm{\mu}$ and $\bm{\Sigma}$, according to the value of the soft-assignment as outlined by algorithm \ref{algorithm:em}.

\begin{equation}
\label{eq:log_likelihood}
p(x) = \sum^K_{k=1} \pi_k \mathcal{N}(x|\bm{\mu}_k, \bm{\Sigma}_k)
\end{equation}


\begin{algorithm}
    \SetAlgoLined
    \KwInput{Set of data vectors X of size M} 
    \KwOutput{K sets of clustered data vectors}
    Initialize the desired number of clusters $K$\;
    Initialise a list of $K$ random cluster centroids $\mu$\;
    \While{$\mu$ elements have not converged}{
        \For{i = 0 to M}
        {
            distOld = $\inf$
            \For{j = 0 to K}{
                distnew = EuclideanDistance(X[i], $\mu[j]$)\;
                \If{distNew $<$ distOld}{
                    distOld = distNew\;
                    $\mu_j$.append(X[i])\;
                }
            }
        }
        \For{p = 0 to K}{
            centroidList.append(average($\mu[p]$))\;
        }
    }
    \caption{K Means Clustering \cite{oreilly_python}}
    \label{algorithm:em}
\end{algorithm}















A binary indicator $z_{i}$ exists for every sample $x_i$, it is 1 if the sample belongs in the cluster $k$ and 0 otherwise such that $z_{ik} \in \{0, 1\}$ and $\sum_k z_{ik} = 1$. If the data is multidimensional then the sample and indicator are both vectors, $\bm{x}$ and $\bm{z}$. The marginal probability $p(\bm{z}_{k} = 1)$ is the probability that a sample $\bm{x}$ is in cluster $k$. This quantity is completely specified by the mixture weight $\pi_k$ for each Gaussian because the area under a Gaussian component is equal to its mixing ratio, i.e.

\begin{equation}
	\label{eq:marginalk}
	p(\bm{z}_{k}=1) = \pi_k 
\end{equation}
	
If we know that a sample $\bm{x}$ is from cluster $k$ the \emph{likelihood} of seeing it in the associated Gaussian is the value of the Gaussian at that point,
\begin{equation}
	\label{eq:cond}
	p(\bm{x}\, |\, \bm{z}_{k} = 1) = \mathcal{N}(\bm{x}\, |\,\bm{\mu_k}, \bm{\Sigma_k})
\end{equation}

The conditional probability of $\bm{z}_{ik}$ given the value of sample, $\bm{x}_i$ is denoted as $\gamma(\bm{z}_{ik})$. This is the probability that a sample belongs to cluster $k$ given its value $\bm{x}_i$ and is the quantity of interest when trying to classify a sample.  Using Bayes Theorem [REF APPENDIX] quantity can be found 

\begin{align}
	\gamma(\bm{z}_{ik}) \equiv p(\bm{z}_{ik} =1 | \bm{x}_i)
	\label{eq:gamma2}
	&= \frac{p(\bm{z}_{ik}=1)p(\bm{x}_i\, |\, \bm{z}_{ik} = 1)}{\sum^K_{j=1}p(\bm{z}_{jk}=1)p(\bm{x}_j\, |\, \bm{z}_{jk} = 1)}\\ 
	\label{eq:gamma3}
	&= \frac{\pi_k\, \mathcal{N}(\bm{x}\,|\,\bm{\mu_k},\bm{\Sigma_k})}{\sum_{j=1}^{K}\pi_k\, \mathcal{N}(\bm{x}\,|\,\bm{\mu_k},\bm{\Sigma_k})}
\end{align}


To maximize the likelihood of each sample being in each Gaussian the distribution parameters are modified. This is achieved by taking derivatives of the log of the likelihood function (\ref{eq:likelihood}) with respect to each Gaussian parameter and setting the result to 0 to find local maxima. To take the log of the likelihood function assume all samples in are in an $N \times D$ matrix $\bm{X}$ and the corresponding indicators are in an $ N \times K$ matrix $\bm{Z}$. 

\begin{equation}
\label{eq:log}
ln\,p(X|\bm{\pi}, \bm{\mu}, \bm{\Sigma}) = \sum_{n=1}^N
ln \Bigg\{ \sum_{k=1}^K \pi_k\mathcal(N)(\bm{x}_n|\bm{\mu}_k,\bm{\Sigma}_k)\Bigg\}
\end{equation}







\centerline{}

The EM algorithm is comprised of [X] steps

%% EM STEPS %%
\begin{enumerate}
	\item Generate the initial K Gaussian parameters mean $\bm{\mu}_{k}$, covariance $\bm{\Sigma}_k$ and mixing ratios $\pi_k$ either randomly or informed by a histogram.
	\item Estimate the likelihood sample n was generated by cluster k for all samples. 
	\begin{equation}
		\gamma(\bm{z}_{ik})=\frac{\pi_k\, \mathcal{N}(\bm{x}\,|\,\bm{\mu_k},\bm{\Sigma_k})}{\sum_{j=1}^{K}\pi_k\, \mathcal{N}(\bm{x}\,|\,\bm{\mu_k},\bm{\Sigma_k})}
	\end{equation}

	\item Maximize the Gaussian parameter using derivatives of log likelihood for each parameter.
	\begin{align}	
	\label{eq:mean}
	\bm{\mu}_{k}^{new} &= \frac{1}{N_k}\sum_{n=1}^N \gamma(z_{nk} ) \bm{x}_n \\
	\label{eq:covariance}
	\bm{\Sigma}_k^{new} &=  \frac{1}{N_k}\sum_{n=1}^N\gamma(z_{nk})(\bm{x}_n- \bm{\mu}_k^{new})(\bm{x}_n - \bm{\mu}_k^{new})^T \\
	\label{eq:ratio}
	\pi_k &= \frac{N_k}{N}
	\end{align}
	where \newline
	\[N_k = \sum_{i} z_{ik}\]
	\item Repeat steps 2 and 3 until convergence of log likelihood (\ref{eq:likelihood}) or parameters. 
\end{enumerate}

The Gaussian Mixture Model method of clustering is computationally complex compared to K-Means however it's ability to differentiate ambiguous samples by considering covariance is superior. 


