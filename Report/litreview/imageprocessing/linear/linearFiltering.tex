\subsection{Linear Filtering}

Linear filtering of an image is an interaction between a filter and a digital image. The output of this operation is the result of the weighted average of the image pixels in a neighborhood specified by the location and size of the filter. The mathematical operation that describes the weighted average produced by the interaction of a linear filter with an image is called convolution (see sections \ref{subsection:corr} \& \ref{subsection:conv}). A filter is characterised by its \emph{kernel}, K, which is comprised of coefficients, or weights, that affect the output of the filter. Convolution multiplies each of the filter's coefficients with its corresponding pixel, sums the results of each multiplication and stores it at the location matching the filter's origin in the output image. Figure \ref{fig:kernel_graphics} visualizes a single kernel application where the filter placed over a neighborhood of pixels in a digital image. The reference pixel is the location for which the result is stored in the output image, to perform this operation on the entire image the process is repeated with the reference pixel centered over every pixel in the image \cite{alg_apps}. Figure \ref{fig:generalForm} shows the general form of a linear filter kernel and that the divisor that determines the value of the result is the sum of all coefficients in the filter kernel.

\begin{figure}[h]
   \[K  = \frac{1}{\sum\limits_{i=0}^{M}\sum\limits_{j=0}^{N}c_{i,j}}
    \begin{bmatrix}
      c_{0,0} & c_{0,1} & \dots & c_{0,n} \\
      c_{1,0} & c_{1,1} & \dots & c_{1,n} \\
      \vdots & \vdots & \ddots & \vdots \\
      c_{m,0} & c_{m,1} & \dots & c_{m,n}
    \end{bmatrix}
  \]
  \caption{General form of a linear filter.}
  \label{fig:generalForm}
\end{figure}

\begin{figure}[H]
  \centering
  \centering\includegraphics[width = 0.8\textwidth]{litreview/imageprocessing/linearfiltering/kernel_graphics}
  \caption{Visualization of a filter kernel application.}
  \label{fig:kernel_graphics}
\end{figure}

% CORRELATION
\subsection{Correlation}
\label{subsection:corr}
The application of a linear filter $h(u,v)$ to an image $f(i,j)$ may be described as follows

\begin{equation} \label{eq:1}
g(i,j) = \sum_{u=-k}^{k}\sum_{v = -l}^{l}f(i+u,j+v)h(u,v)
\end{equation}

$g(i,j)$ is the output image. Performing correlation with a filter may be notated more concisely by the correlation operator.

\[g = f \otimes h\]

Correlation measures the similarity between two signals. Both digital images and linear filters are two dimensional signals. Performing correlation between them will yield an output image where the highest values correspond to where the image and filter were most similar \cite{optimalKernel}. This useful because if you wish to emphasise a feature in an image it can be done by correlating it with a filter that describes that feature. For example, if you wished to exagerate vertical and horizontal lines you could use Sobel filters as in \ref{fig:sobel_filters}. 

% SOBEL MASKS
\begin{figure}[H]
  \begin{subfigure}[b]{0.49\textwidth}
    \[
    \begin{bmatrix*}[l]
     -1 & -1 & -1 \\
      \phantom{-}2 & \phantom{-}2 & \phantom{-}2 \\
      -1 & -1 & -1 
    \end{bmatrix*}
    \]
    \caption{Horizontal Sobel Filter Mask}
    \label{rfidtest_xaxis}
\end{subfigure}
\begin{subfigure}[b]{0.49\textwidth}
  \[ 
    \begin{bmatrix}
      -1 & 2 & -1 \\
      -1 & 2 & -1 \\
      -1 & 2 & -1
    \end{bmatrix}
    \]
    \caption{Vertical Sobel Filter Mask}  
\end{subfigure}
    \caption{Sobel Filters}
    \label{fig:sobel_filters}
\end{figure}

The use of negative weightings means that values next to an edge are diminished and the positively weighted line sections (the 2s) strengthen line features. Notice in figure \ref{fig:sobel_apply} how the lines have a high values (white) and regions that aren't lines are low valued (black).

% SOBEL FILTER APPLICATION
\begin{figure}[H]
  \centering
  \begin{subfigure}[b]{0.3\textwidth}
      \includegraphics[width=\textwidth]{im_color}
      \caption{Image by Simone Hutsch}
  \end{subfigure}
  \begin{subfigure}[b]{0.3\textwidth}
      \includegraphics[width=\textwidth]{gv}
      \caption{Vertical Sobel Filter}
      \label{fig:vert}
  \end{subfigure}
  \begin{subfigure}[b]{0.3\textwidth}
      \includegraphics[width=\textwidth]{gh}
      \caption{Horizontal Sobel Filter}
      \label{fig:hoz}
  \end{subfigure}
  \caption{Application of Sobel filters to exagerate lines.}
  \label{fig:sobel_apply}
\end{figure}

Correlation is \emph{shift invariant}, which means that it does the same thing no matter where in an image it is applied. To satisfy this property correlation may be superpositioned 

\[a(f_1 + f_2) = af_1 + af_2\]

and abides by the shift invariance principle

\[g(i,j)=f(i+k,j+l) \Leftrightarrow\ (h\circ g)(i,j)=(h\circ f)(i+k,j+l)\]

Correlation has the side effect of flipping both horizontally and vertically the location of output points relative to location the center point (\emph{reference point}) in the original image which may be undesirable.

% CORRELATION EXAMPLE  %
\begin{figure}[H] 
  \centering
  \begin{tabular}{ccccc}
      \begin{tabular}{|c|c|c|c|c|}
      \hline
      0 & 0 & 0 & 0 & 0 \\[1pt]
      \hline
      0 & 0 & 0 & 0 & 0 \\[1pt]
      \hline
      0 & 0 & 1 & 0 & 0 \\[1pt]
      \hline
      0 & 0 & 0 & 0 & 0 \\[1pt]
      \hline
      0 & 0 & 0 & 0 & 0 \\[1pt]
      \hline
      \end{tabular}%
    & $\otimes$ &
    \begin{tabular}{|c|c|c|}
      \hline
      a & b & c \\
      \hline
      d & e & f \\
      \hline
      g & h & i \\
      \hline 
    \end{tabular}
    & $=$ &
    \begin{tabular}{|c|c|c|c|c|}
      \hline
      0 & 0 & 0 & 0 & 0 \\[1pt]
      \hline
      0 & \textbf{i} & \textbf{h} & \textbf{g} & 0 \\[1pt]
      \hline
      0 & \textbf{f} & \textbf{e} & \textbf{d} & 0 \\[1pt]
      \hline
      0 & \textbf{c} & \textbf{b} & \textbf{a} & 0 \\[1pt]
      \hline
      0 & 0 & 0 & 0 & 0 \\[1pt]
      \hline
    \end{tabular} \\
    $F(x,y)$ & & $H(u,v)$& & $G(x,y)$ \\
  \end{tabular}
  \caption{Correlation of a filter and an image.}
  \label{fig:correlation}
\end{figure}
\subsection{Convolution}
\label{subsection:conv}
Convolution is also a linear operation that is shift invariant. It is very similar to correlation except that where correlation measure similarity between signals convolution measures the effect of one signal on another. It is described mathematically by the expression,

\[ g(i,j) = \sum_{u=-k}^{k}\sum_{v = -l}^{l}f(u,v)h(i-u,j-v)\]

Notice that the filter $h(i,j)$ is rotated  180 degrees. This causes the output's orientation to match the original image. Convolution may be notated as follows,

\[g = f \ast h\]

Convolution is essentially the same as correlation except that it doesn't flip the output relative to the original image as can be seen in Figure \ref{fig:convolution} and it is the operation that is performed when a linear filter is applied to a digital image.

% Convolution example  %
\begin{figure}[H] 
  \centering
  \begin{tabular}{ccccc}
      \begin{tabular}{|c|c|c|c|c|}
      \hline
      0 & 0 & 0 & 0 & 0 \\[1pt]
      \hline
      0 & 0 & 0 & 0 & 0 \\[1pt]
      \hline
      0 & 0 & 1 & 0 & 0 \\[1pt]
      \hline
      0 & 0 & 0 & 0 & 0 \\[1pt]
      \hline
      0 & 0 & 0 & 0 & 0 \\[1pt]
      \hline
      \end{tabular}%
    & $\ast$ &
    \begin{tabular}{|c|c|c|}
      \hline
      a & b & c \\
      \hline
      d & e & f \\
      \hline
      g & h & i \\
      \hline 
    \end{tabular}
    & $=$ &
    \begin{tabular}{|c|c|c|c|c|}
      \hline
      0 & 0 & 0 & 0 & 0 \\[1pt]
      \hline
      0 &  \textbf{a} & \textbf{b} & \textbf{c} & 0 \\[1pt]
      \hline
      0 & \textbf{d} & \textbf{e} & \textbf{f} & 0 \\[1pt]
      \hline
      0 & \textbf{g} & \textbf{h} & \textbf{i} & 0 \\[1pt]
      \hline
      0 & 0 & 0 & 0 & 0 \\[1pt]
      \hline
    \end{tabular} \\
    $F(x,y)$ & & $H(u,v)$& & $G(x,y)$ \\
  \end{tabular}
  \caption{Convolution of a filter and an image.}
  \label{fig:convolution}
\end{figure}
\subsection{Kernels}

Kernels are just the weightings that define the characteristics of a filter, also known as a filter's mask. A filter kernel is nearly always square so as to have a center cell which sits atop a reference pixel. The result of the filter's application at that reference pixel will be stored in the output image at the location of the reference pixel. Notice in Figure \ref{fig:kernel_graphics} how the mask sits over the reference pixel.

% TREE %
\begin{figure}[H]
 \centering
 \centering\includegraphics[width=350pt]{kernel_graphics}
 \caption{Visualization of a Filter Kernel Application}
 \label{fig:kernel_graphics}
\end{figure}

\subsubsection{Box Filter}
\label{subsubsection:boxfilter}
A box filter also known as a moving average filter simply outputs the average of its inputs because the filter weights are evenly distributed. By passing this filter over an image its sharpness is reduced giving a smoothing or blurring effect which can sometime be useful in image processing for filtering out noise. This can be observed in Figure \ref{fig:roughDog}.

% DOG BOX FILTERED%
\begin{figure}[H]
 \centering
 \begin{subfigure}[b]{0.75\textwidth}
   \centering\includegraphics[width=300pt]{dogPug}
   \caption{Left: Photo by Charles Deluvio. Right: Application of $16\times16$ box filter.}
   \label{fig:roughDog}
 \end{subfigure}
 \begin{subfigure}[b]{0.25\textwidth}
   \centering
   \[
     \frac{1}{9}
   \begin{bmatrix}
      1 & 1 & 1 \\
      1 & 1 & 1 \\
      1 & 1 & 1
   \end{bmatrix}
   \]
   \caption{Box Filter Kernel.}
   \label{fig:boxkernel}
 \end{subfigure}
 \caption{Box Filter application and kernel.}
 \label{fig:boxfilter}
\end{figure}

\subsubsection{Gaussian Kernel}

The Gaussian is a rather important kernel in image processing as it models the Gaussian function (see Section \ref{section:gaussian}) or normal distribution. This filter is perhaps most well known for filtering noise but retaining edge sharpness better than other denoising filters. It works in a similar fashion to the moving average filter but as you can see in Figure XXX it is superior at filtering out the noise in comparison as it retains better edge definition. The reason the Gaussian filter is superior is because it addresses two properties about images that a generally true, 
\begin{enumerate}
    \item The actual value of a pixel is probably the same or similar to its neighbors. 
    \item Each pixel of noise in an image is added independently.
\end{enumerate}

The Gaussian which models the Normal distribution addresses both of these qualities by having high value cooefficients at the filter's center and lower values tapering out to the edges of the filter. Which essentially means values closer together are more strongly correlated than those further away from the reference pixel as can be observed in the 16x16 Gaussian filter kernel in Figure XXX.

[ Add Denoising using Gaussian Example ]
[ Add matrix of kernel filter. ]






