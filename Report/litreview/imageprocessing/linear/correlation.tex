% CORRELATION
\subsubsection{Correlation}
\label{subsection:corr}

The correlation of a linear filter $h(u,v)$ to an image $f(i,j)$ with the output $g(i,j)$ may be described mathematically as in Equation \ref{eq:1} but denoted more succinctly as in Equation \ref{eq:correlation_operator}

\begin{equation} \label{eq:1}
g(i,j) = \sum_{u=-k}^{k}\sum_{v = -l}^{l}f(i+u,j+v)h(u,v)
\end{equation}

\begin{equation} \label{eq:correlation_operator}
g = f \otimes h
\end{equation}

Correlation measures the \emph{similarity} between two signals and is used in image processing to measure the similarity between digital images and linear filters. When correlating an image and a linear filter, regions of high similarity will produce high output values \cite{optimalKernel}, this phenomenon can be used to isolate features in an image by correlating a filter that describes the desired feature. A feature could be as simple as a straight line as in a Sobel filter (Figure \ref{fig:sobel_filters}) or as specific as the outline of a car wheel. This method of feature detection is called \emph{template matching} \cite{oreilly_python}. In Figure \ref{fig:sobel_apply} vertical and horizontal Sobel filters \cite{cv_matlab} are correlated with an image a building wall and produce images that emphasize lines that match the orientation of the respective filter. The result is a binary image the output has been thresholded (see section \ref{subsection:thresholding}) to isolate high intensity regions where similarity is greatest.

% SOBEL MASKS
\begin{figure}[H]
  \begin{subfigure}[b]{0.49\textwidth}
    \[k =
    \begin{bmatrix*}[l]
     -1 & -1 & -1 \\
      \phantom{-}2 & \phantom{-}2 & \phantom{-}2 \\
      -1 & -1 & -1 
    \end{bmatrix*}
    \]
    \caption{Horizontal Sobel-Feldman Filter Mask}
    \label{rfidtest_xaxis}
\end{subfigure}
\begin{subfigure}[b]{0.49\textwidth}
  \[ k = 
    \begin{bmatrix}
      -1 & 2 & -1 \\
      -1 & 2 & -1 \\
      -1 & 2 & -1
    \end{bmatrix}
    \]
    \caption{Vertical Sobel-Feldman Filter Mask}  
\end{subfigure}
    \caption{Sobel Filters}
    \label{fig:sobel_filters}
\end{figure}

% SOBEL FILTER APPLICATION
\begin{figure}[H]
  \centering
  \begin{subfigure}[b]{0.3\textwidth}
      \includegraphics[width=\textwidth]{im_color}
      \caption{Grayscale image. Img: Simone Hutsch}
  \end{subfigure}
  \begin{subfigure}[b]{0.3\textwidth}
      \includegraphics[width=\textwidth]{gv}
      \caption{Binary vertical similarity.}
      \label{fig:vert}
  \end{subfigure}
  \begin{subfigure}[b]{0.3\textwidth}
      \includegraphics[width=\textwidth]{gh}
      \caption{Binary horizontal similarity.}
      \label{fig:hoz}
  \end{subfigure}
  \caption{Application of Sobel filters to exaggerate lines.}
  \label{fig:sobel_apply}
\end{figure}

Correlation is \emph{shift invariant}, which means that it does the same thing no matter where in an image it is applied. To satisfy this property correlation may be super-positioned 

\[a(f_1 + f_2) = af_1 + af_2\]

and abides by the shift invariance principle

\[g(i,j)=f(i+k,j+l) \Leftrightarrow\ (h\circ g)(i,j)=(h\circ f)(i+k,j+l)\]

Correlation has the side effect of flipping both horizontally and vertically the location of output points relative to the center point (\emph{reference point}) in the original image which may be undesirable as can be observed in Figure \ref{fig:correlation}.

% CORRELATION EXAMPLE  %
\begin{figure}[H] 
  \centering
  \begin{tabular}{ccccc}
      \begin{tabular}{|c|c|c|c|c|}
      \hline
      0 & 0 & 0 & 0 & 0 \\[1pt]
      \hline
      0 & 0 & 0 & 0 & 0 \\[1pt]
      \hline
      0 & 0 & 1 & 0 & 0 \\[1pt]
      \hline
      0 & 0 & 0 & 0 & 0 \\[1pt]
      \hline
      0 & 0 & 0 & 0 & 0 \\[1pt]
      \hline
      \end{tabular}%
    & $\otimes$ &
    \begin{tabular}{|c|c|c|}
      \hline
      a & b & c \\
      \hline
      d & e & f \\
      \hline
      g & h & i \\
      \hline 
    \end{tabular}
    & $=$ &
    \begin{tabular}{|c|c|c|c|c|}
      \hline
      0 & 0 & 0 & 0 & 0 \\[1pt]
      \hline
      0 & \textbf{i} & \textbf{h} & \textbf{g} & 0 \\[1pt]
      \hline
      0 & \textbf{f} & \textbf{e} & \textbf{d} & 0 \\[1pt]
      \hline
      0 & \textbf{c} & \textbf{b} & \textbf{a} & 0 \\[1pt]
      \hline
      0 & 0 & 0 & 0 & 0 \\[1pt]
      \hline
    \end{tabular} \\
    $F(x,y)$ & & $H(u,v)$& & $G(x,y)$ \\
  \end{tabular}
  \caption{Correlation of a filter and an image.}
  \label{fig:correlation}
\end{figure}