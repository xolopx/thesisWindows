\section{Clustering}
Clustering is a method of segmenting an image into disjoint sets known as classes. This is useful for indentifying types of features or objects in an image. For example in Figure \ref{fig:toycar} the toy car has been identified out of the background.l

\begin{figure}[H]
	\centering
	\begin{subfigure}[b]{0.5\linewidth}
      		\centering\includegraphics[width=220pt]{toyCar}
      		\caption{Image by Gustavo, Upsplash.}
		    \label{fig:toycarA}
    	\end{subfigure}%
    	\begin{subfigure}[b]{0.5\linewidth}
      		\centering\includegraphics[width=220pt]{toycarseg}
      		\caption{Segmentation of car from background.}
       		\label{fig:toycarB}
    	\end{subfigure}
    	\caption{Segmentation of toy cars using K-Means Clustering.}
    	\label{fig:toycar}
\end{figure} 

In the context of an image generally every pixel is individually classified however groups of pixels known as superpixels may also be classified. Entities are classified based on the similarity of their features. For example, a pixel's intensity is a feature that could be used to cluster it with other pixels of similar intensity. As seen in (\ref{eq:featurevector}) an entity may be represented by any number of features, known also as dimensions, $d$. 

\begin{equation}
    \vec{v} = 
    \begin{bmatrix}
        d_1 \\
        d_2 \\
        \vdots \\ 
        d_n
    \end{bmatrix}
    \label{eq:featurevector}
\end{equation}



\subsection{K-Means}
K-Means is simple and reasonably fast clustering algorithm. K represented the number of cluster the algorithm should create and means refers to the average value of each cluster. Given a set of data their similarity is measured as the value of the Euclidean distance between them. The euclidean distance formula is described for two dimensional data in equation \ref{eq:euclid}, this would be suitable for spatial data in a 2D plane such as an image, whose pixels are located in the dimension x and y. As the K-means algorithm executes, data samples are assigned to the cluster whose average value is closest to its own, the centroids are updated after each round of assignments until their values converge. Initially the average value of all clusters is randomly selected, referred to as cluster centroids. The algorithm to implement the K-means method is outlined in algorithm \ref{algorithm:k_means}.

\begin{equation}
    D(\vec{p},\vec{q}) = \sqrt{(p_1 - q_1)^2 + (p_2 - q_2)^2 +\hdots + (p_n - q_n)^2}
    \label{eq:euclid}
\end{equation} 

\begin{algorithm}
    \SetAlgoLined
    \KwInput{Set of data X of size M} 
    \KwOutput{K sets of clustered data}
    Initialize the desired number of clusters $K$\;
    Intialize a list of $K$ random cluster centroids $\mu$\;
    \While{$\mu$ elements have not converged}{
        \For{i = 0 to M}
        {
            $distOld = \inf$
            \For{$j = 0 to K$}{
                $distnew = euclideanDistance(X[i], \mu_j)$\;
                \If{distNew \le distOld}{
                    $distOld = distNew$\;
                    $\mu_j$.append(X[i])\;
                }
            }
        }
        \For{p = 0 to K}{
            centroidList.append(average($cluster_p$))
        }
    }
    \caption{K-Means Clustering \cite{oreilly_python}}
    \label{algorithm:k_means}
\end{algorithm}

The best result for this method is defined as having the smallest intracluster variance. K-Means is disadvantaged by the implicit trait that it formulates clusters of similar sizes. This happens because the algorithm seeks to minimize variance (spread) in each cluster hence the \q{ideal} centroid placement will form distributions spherically about centroids. This method cannot disentangle overlapping samples that belong in different classes. In Figure \ref{fig:clusters} clear edges between clusters and sphereical distributions can be observed. The algorithm can only implement \emph{hard assignments}, as opposed to a \emph{soft assignment} that consider the probability of a sample's class membership. 



\begin{figure}[H]
    \centering
    \centering\includegraphics[width=300pt]{kmeans_clusters}
    \caption{K-Means clustering performed on random data.}
    \label{fig:clusters}
\end{figure} 
\section{Mixture of Gaussians}

Method of modelling each background pixel by a mixture of K Gaussian distributions. 

Each background colour is represented by a Gaussian. A colour is most likely background if it stays static longer. Static single-colour objects form tight clusters in the colour space while moving ones form wide clusters due to different reflecting surfaces during movement. 

\subsection{Background Subtraction}

This is performed by identifying a background image and then subtracting this image from subsequent frames of video to determine what has changed and hence what is a foreground image.

\subsection{Expectation Maximization}

This is an optimization scheme used to fit a Gaussian Mixture Model. It is an iterative method that faurantees convergence to a local maximum in a search space.
