\chapter{Conclusion}

\section{System Outcomes}

The objective of the system, to create lightweight traffic data collection system, was satisfied. The system achieved accuracies of 95\% and 95\% for volume and speed measurements respectively, satisfying the primary objective of collecting traffic data. The best results were obtained for camera setups looking down at a road scene as this angle provided the least vehicle occlusion. The system was implemented on versatile and inexpensive hardware showing that a lightweight and modular data collection system is plausible and appropriate for a Smart City. The system's detection algorithm was shown to suffer accuracy reductions due to edge cases caused by camera movement and partial or complete vehicle occlusion. Handling these cases requires additional complexity in the computer vision subsystem that may comprise the system's simplicity and reduce its execution speed. That the system produced real-time statistics extends its usability to applications that require real-time data such as live traffic rerouting. 

\section{System Extensions}

There are a number of additions and improvements that could be made to the system to improve its overall performance and the experience of those using the system. These additions were omitted from the system's present implementation due to time restrictions. 

The most beneficial extension would be to test and implement node networking on a wide area network such as 4G as this iteration of the system was implemented only on a local area network. Communication with nodes over long distances would enable technicians to calibrate them without having to go to the installation site reducing costs. Node calibration could be added to the user interface and show in real-time the effect of modifying detection algorithm parameters. 

To address the edges cases such as vehicle occlusion new features could be considered in the detection algorithm such as object luminance and vehicle shape. This additional information could be used to verify not only that an object was a vehicle but also more accurately develop the object's shape and hence determine the \emph{type} of the vehicle (truck, car, motorbike...) which is valuable traffic data in and of itself. 

This iteration of the system was tested very little under nighttime conditions and so this is a natural extension of the project to provide a system that develops a whole picture of a traffic network's usage. This could be implemented with many of the same techniques that are already present in the system but with the addition of a light sensor to determine when it's dark and an infra-red camera to capture images. 