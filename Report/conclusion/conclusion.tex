\chapter{Conclusion}

\section{System Outcomes}

The objective of the project, to create lightweight traffic data collection system, was satisfied. The system achieved an average and median F1 value of 94.40\% and 96.00\% respectively for traffic volume measurement tests conducted for seven non-optimal traffic scenarios. The optimal camera setups was found to be looking directly down at a road as this angle provided the least vehicle occlusion. The system was implemented on versatile and inexpensive hardware showing that a lightweight real-time data collection system is plausible and appropriate for a Smart City. The system's detection algorithm was shown to suffer accuracy reductions due to edge cases caused by camera movement and partial or complete vehicle occlusion. Handling these cases requires additional complexity in the computer vision algorithm that may comprise the system's simplicity and reduce its execution speed, though these issues can mostly be avoided using the recommended camera setup. That the system produced real-time statistics extends its usability to applications that require real-time data such as live traffic rerouting. 

\section{System Extensions}

There are a number of additions and augmentations that could be made to the system to improve its overall performance and the experience of those using the system. These additions were omitted from the system's present implementation due to time restrictions. 

The most beneficial extension would be to test and implement node networking on a wide area network such as 4G, enable remote communications across the internet. Access to a telephone network would require an additional networking chip on the Raspberry. Communication with nodes over the internet would enable technicians to calibrate them without having to go to the installation site reducing costs. Node calibration could be added to the web-app interface and show in real-time the effect of modifying detection algorithm parameters. Combining wireless network capability with an off-grid power source such a solar panel and battery would mean that the system could be installed without needing to provide mains power furthering it's modularity and versatility.

To address the edges cases such as vehicle occlusion new features could be considered in the detection algorithm such as object luminance and vehicle edges. This additional information could be used to verify not only that an object was a vehicle but also more accurately derive a vehicle's shape thus enabling  the \emph{type} of the vehicle to be determined.

This iteration of the system was tested very little under night-time conditions and so this is a natural extension of the project to provide a system that can effectively gather data all the time. This could be implemented with many of the same techniques that are already present in the system but with the addition of an infra-red camera to capture images at night-time. 

Finally measurement of vehicle speed would provide the second dimension of the data picture of a traffic network. This functionality was almost implemented but could not tested in satisfactory manner due to time constraints and for non-optimal camera setups yielded low accuracy data. An accurate implementation for non-optimal camera setups would require two cameras to conduct photogrammetry that would allow the depth of an object to be determined. 