\chapter{Introduction}
\parskip 0.2in
 Economies of the world have been industrializing for over two centuries \cite{industrialization} and the consequent advent of machines and factories has seen a great migration of humanity from rural fields to urban habitats. Such a concetration of people has posed plethora of infrastructural challenges. This has lead to many iconic solutions to high density living, like the skyscraper, subway train and traffic light. One unfortaunte metropolitan trait, however, continues unabated and is in fact growing in size, it is traffic congestion \cite{trafficworse}.  

Road transport contributes 16.5\% of global CO2 emissions \cite{emissions} and costs the US \$305 billion of productivity per year alone \cite{ecotoll} \cite{cost}. Any small improvement in this situation will yield great benefit to society and the future of the planet.  

There is no single solution to the problem of traffic congestion, whether it be the building of tunnels, the conversion of traffic lights to roundabouts, the widening of roads or the narrowing of roads. The integration of several lightweight and inexpensive techniques is a far more attractive and effective approach \cite{mixedSolution}. A ‘Smart City’ is such an approach, it is implemented by deploying many IoT devices in an urban area to collect data \cite{iot}. This technology could be applied to collect traffic data and use it to inform investment in a traffic network. 

A versatile and inexpensive method of data collection in a traffic network is computer vision. This technique requires as input only raw images of a traffic network and could therefore piggyback onto existing surveillance cameras. The intention of this technology is to mimic the ability of human vision to rapidly identify distinct objects from the real world. A human’s ability to separate a vehicle from its surroundings and track its movement is one example of this.  

The implementation of a robust computer vision algorithm for traffic monitoring is not trivial. The dependence on only one feature in an image will not yield accurate results, however the combination of several different features will, features like object edges, object area, hue, contrast and reflectance. There are a great many and varying number of ways in which to extract information from these features. Whatever method of implementation is selected must provide consistent accuracy, and in the case of traffic monitoring, real time results.  

Therefore, it is the objective of this report to explain the design and operation of a novel computer vision algorithm that can effectively count and estimate the speed of vehicles in traffic, in real-time, given raw images of a traffic network. 



