\subsection{Tracking}

The tracking component of the detection algorithm takes a list of bounding box coordinates corresponding to each foreground object and generates tracked objects that contain the coordinates history of the objects. 

Each vehicle object's location in an image is represented by the centre of its bounding box, called a centroid, by maintaining a history of an object's centroid position over time its path can be tracked through the video. For each image new centroids are recalculated for every object and by comparing the new centroid locations with the old centroid locations the new position of objects can be determined. New centroids are assigned to the objects whose last known position was closest. Algorithm \ref{algorithm:centroid_reassign} outlines the centroid reassignment process.

\begin{algorithm}
  \SetAlgoLined
  \KwInput{List of current centroids \emph{cen\_old}, size X, List of new centroids \emph{cen\_new}, size Y} 
  \KwOutput{Updated list of current centroids}
  Generate an $X \times Y$ distance matrix, $D$, between \emph{cen\_old} and \emph{cen\_new}\;
  Generate a list of column indices, \emph{cols} from $D$ sorted by the column containing the smallest value to the largest\;
  Generate a list of row indices, \emph{rows}, from $D$ corresponding to the row containing the smallest values in \emph{cols}.\;
  \For{(i,j) in (\emph{rows}, \emph{cols})}{
    \If{cen\_new[j] not used and cen\_old[i] not used}{
        \emph{cen\_old[i]} = \emph{cen\_new[j]}\;
        mark \emph{cen\_old[i]} as used\;
        mark \emph{cen\_new[j]} as used\;
    }
  }
  \For{\emph{cen} in unused \emph{cen\_new}}{
    \emph{cen\_old}.append(\emph{cen})\;
  }
  \For{\emph{cen} in unused \emph{cen\_old}}{
    mark \emph{cen} as missing\;
  }
  \caption{Centroid re-assignment algorithm. \cite{adrian_rosebrock_simple_object_tracking}}
  \label{algorithm:centroid_reassign}
\end{algorithm}

If a new position cannot be found for an object then it has a limited number of frames it can be missing for before it's removed, objects generally go missing due to partial or complete occlusion by other vehicles, alternatively they will go missing permanently if they leave the node's field of view. Vehicle's can also disappear if they remain stationery long enough to be considered background pixels by the subtractor. Figure \ref{fig:centroids} visualizes the centroids on the objects along with a unique identifier which is used to mark ownership of a foreground object. Notice in Figure \ref{fig:centroids} centroids 25 and 24 belong to no object, this is because they have not yet timed out, meaning their object hasn't been missing from frame long enough for them to be dismissed, this is also the case for centroid 16. Centroids 1 and 9 belong to a cluster of vehicles that have become distant (see Figure \ref{fig:original_frame}), this occurs because as the vehicles travel further away the distance between them in pixels becomes less and so the morphological process presses them together. This is not an issue if the counting and measuring system is calibrated to focus on areas where vehicles are easily separable. 

\begin{figure}[H]
    \centering
    \centering\includegraphics[width = 0.8\textwidth]{design/detection/tracking/mask_centroids}
    \caption{Centroids plotted over foreground objects.}
    \label{fig:centroids}
\end{figure}

\subsubsection{Calibration}

For a given traffic scenario we seek to set a maximum\_distance and minimum\_distance which control centroid reassignment. We also should consider the amount of time a object can go missing for. 

The minimum\_distance parameter addresses failure of the morphology component of the algorithm to recombine some blobs. After all centroids are reassigned an additional consolidation process is performed where if any two centroids are closer together than min\_distance they are merged into a single centroid. Figure \ref{fig:centroids_consolidation} compares two centroid allocations for the same image where one performs centroid combination and the other doesn't. In Figure \ref{fig:consolidateA} the number of centroids is greater thus consuming more computational power and increasing the number of false positive counts.

Maximum\_distance controls the allowed distance a new centroid can be from an old one and still be considered its new position. In cases where there are more new centroids than old ones, without this parameter restriction, centroid allocations will store an old centroid's an unreasonable distance from its old position when in reality a new centroid should be generated and not reassignment performed.

The second calibration consideration for tracking is how long the system allows an object to be missing. Determination of this quantity dependent on how frequently vehicles are occluded and for how long. In a situation where there was an obstacle consistently blocking vision of the road for a period then the amount of time a centroid should be stored is proportional to how long a vehicle is occluded by that obstacle on average. Its common that an object may disappear for a few frames so it's important for objects to be able to go missing and remerge else lots of data could be lost.

\begin{figure}[H]
\centering
\begin{subfigure}[b]{0.45\linewidth}
            \centering\includegraphics[width = \textwidth]{design/detection/calibration/mask_noconsolidate}
            \captionsetup{format=hang}
            \caption{Centroid tracking with no consolidation distance set (16 centroids).}
            \label{fig:consolidateA}
  \label{fig:}
    \end{subfigure}
    \begin{subfigure}[b]{0.45\linewidth}
            \centering\includegraphics[width = \textwidth]{design/detection/calibration/mask_consolidate}
            \captionsetup{format=hang}
        \caption{Centroid tracking with consolidation distance set (20 centroids).}
        \label{fig:consolidateB}
      \end{subfigure}
      \captionsetup{format=hang}
    \caption{Comparison of centroid generation with and without consolidating centroids based on distance.}
    \label{fig:centroids_consolidation}
\end{figure}
  
