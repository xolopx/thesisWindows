\section{Computer Vision}

The computer vision subsystem is a software component that is designed to recognise and track vehicles in images in order to count them and estimate their speed, henceforth to be referred to as the \emph{detection} component of the system. Each time an image is processed by the detection algorithm several subprocesses are performed in order to manipulate the original image into a state where vehicles have been isolated. These subprocesses can be seen in Figure \ref{fig:detection_design}. The vehicle tracking subprocess was largely influenced by the work of Adrian Rosebrock \cite{adrian_rosebrock_simple_object_tracking}\cite{adrian_rosebrock_vehicle_tracking} of PyImageSearch and the vehicle segmentation subprocess was influenced by the work of Andrey Nikishaev \cite{andrey_nikishaev_traffic_counting}. Much of the functionality is provided by the Python open source computer vision library, OpenCV \cite{opencv}


\begin{figure}[H]
    \includegraphics[width=0.9\columnwidth]{design/detection/detection_design}
    \caption{Subprocess that comprise the detection subsystem.}
    \label{fig:detection_design}
\end{figure}



\subsection{Countours and Bounding Boxes}

Contours represent the outline of a foreground object and bounding boxes are the corrected squares that are placed around the polygons that represent an object's outline. These bounding boxes are important for in the tracking of a object.




\subsection{Tracking}

Tracking is follows an objects movement across multiple images. This requires that the object be reidentified in each new image and matched to it's old location.

\subsection{Speed Estimation and Counting}

This subprocess consolidates the efforts of the former ones and convert the information about an objects position into vehicle counts and speeds.


\subsection{Background Subtraction}
\label{subsection:training}
Background subtraction is the first step is determining which pixels of an image belong to a vehicle and divides all pixels in an image into two groups, foreground and background. This background subtraction implementation uses a clustering method known as a Mixture of Gaussian or Gaussian Mixture Model (\ref{subsection:mog}). 

Background subtraction is the most complex and thus the most computationally expensive part of the detection process. It is imperative, however, that the algorithm be fast enough to process images in real-time, such that traffic data is reflect live conditions. Fortunately the OpenCV's BackgroundSubtractorMOG2 function \cite{opencv_mog2} is both fast and effective making it a suitable choice for a low power microcontroller. OpenCV's implementation is based on the the research by Zoran Zivkovic and Ferdinand van der Heijden \cite{zivkovic_pattern_recognition} \cite{zivkovic_heijden_pattern_recognition_letters} which takes only pixel colours as input.

Figure \ref{fig:example_subtraction} shows the application of the background subtractor on an arbitrary traffic image highlighting its ability to isolate, in particular, moving objects. Movement changes the value of pixels over a short time scale making it easy for the model to detect them as foreground objects. Shadows also register as a foreground objects but fortunately OpenCV's implementation is able to explicitly detect shadows and present them as gray pixels in the resulting foreground mask. In Figure \ref{fig:foreground_mask_filtered} the gray pixels in and around the white foreground objects are shadows which are removed via thresholding (equation \ref{eq:threshold}). 

\subsubsection{Calibration}

OpenCV's Gaussian Mixture Model (GMM) implementation has two important parameters to consider, \emph{history} and \emph{shadows}, where history is an integer and shadows is a Boolean. 

The history parameter determines how many formerly processed images affect the Gaussian Mixture Model, for example, if the subtractor's history is 500 and it processes 501 images then the 1st image no longer has influence on the subtractor's models but the 501st does. For a 30fps video 500 images is only about 16 seconds, therefore if a vehicle stops for that long its presence will be completely reflected in the cluster models. In Figure \ref{fig:cluster_adaptation} the vehicles in the third lane are stationary for sufficiently long to be absorbed into the background by the GMM, this is not an issue as when they move again they are recognized once more. Thus, the history length should be long enough to absorb small but consistent changes in pixel values, like swaying tree branches, into the background but short enough that a slowly moving vehicle registers as a foreground object. In short the longer the history the more insensitive the background model becomes to change.

The shadows parameter tells the model if a cluster for shadows should be specified, if they are they appear as gray, not white, in the foreground mask. In this system's implementation they are set to true as it allows them to be removed from the foreground mask, Figures \ref{fig:foreground_mask_unfiltered} and \ref{fig:thresh_shadow} show their presence and consequent removal via thresholding. 

Finally, the subtractor when initialized hasn't had a chance to develop cluster models for the given traffic scene resulting foreground mask with a lot of anomalies. By 'training' the model on a few seconds of video before using it to collect traffic data any initial erroneous readings are avoided. This simply involves feeding the algorithm a few hundred images before using it to collect data.

\begin{figure}[H]
	\centering
	\begin{subfigure}[b]{0.4\linewidth}
        \centering\includegraphics[width = \textwidth]{design/detection/calibration/original}
        \caption{Original Frame - vehicles moving.}
    \end{subfigure}%
    \begin{subfigure}[b]{0.4\linewidth}
        \centering\includegraphics[width = \textwidth]{design/detection/calibration/mask}
        \caption{Foreground Mask - vehicles moving.}
    \end{subfigure}
    \begin{subfigure}[b]{0.4\linewidth}
        \centering\includegraphics[width = \textwidth]{design/detection/calibration/original_adapted}
        \caption{Original Frame - vehicles stopped.}
    \end{subfigure}%
    	\begin{subfigure}[b]{0.4\linewidth}
        \centering\includegraphics[width = \textwidth]{design/detection/calibration/mask_adapted}
        \caption{Foreground Mask - vehicles stopped.}
    \end{subfigure}
    \caption{Adaptation GMM background cluster. Images by Karol Majek}
    \label{fig:cluster_adaptation}
\end{figure}



\subsection{Morphology and Filtering}

After obtaining the foreground mask it's necessary to remove noise, consolidate certain 'blobs' and discard others. The GMM clusters pixels on their colour and how much it deviates from the accepted background model, thus darker pixels, in car windows in particular, can be classified as background pixels resulting into a series of disconnected components where only a single connected component should exist. In Figure \ref{fig:example_subtraction} a number of small foreground blobs often appear where a single vehicle exists in the corresponding original image, by applying morphology to the mask these blobs can be consolidated.

\subsubsection{Filtering}

Often salt and pepper noise can result from the subtraction process due to spurious lighting changes but by applying a median filter (\ref{subsection:median_filter}) it can be removed. Figure \ref{fig:foreground_mask_filtered} shows the effect of applying a median filter to remove noise from a foreground mask. It's important to remove this noise because firstly it's not a vehicle and secondly subsequent morphological operations may accentuate the noise and cause the system to generate incorrect output.


\subsubsection{Morphology}

Morphological operations are required to consolidate the foreground blobs that are disconnected when they should in fact be singular completely connected entities. This is important for the contouring process where an object is recognised by it's size and if a vehicle's foreground mask consists of a series of small disconnected blobs it will not be recognised. To consolidate these lonely blobs into a happy whole a morphological closing and dilation (\ref{section:morphology}) are performed the effect of which can be observed in Figure \ref{fig:foreground_mask_morphed}. The selection of structuring element shapes, size and number of iterations is dependent on the specific traffic scene presented and requires considerable calibration. 

\subsubsection{Calibration}

Generally, in at least one region of the foreground mask a superior formation of connected components straight out of the subtractor exists but there will always remain a number of disconnected components that require consolidation. By focusing on improving the consolidation of blobs in the aforementioned superior area this region can be exploited, furthermore, the counting and speed measuring operate only in a specific region where recognition and tracking is best. The reason that a single area must be selected and that object detection performance varies across an image is because of the camera's perspective of the traffic environment. If a camera is setup looking along the direction of movement of vehicles then their perceived size of said vehicles changes as they move closer and further from the camera. This affects numerous factors in the detection process but in the case of morphology it means that the features of vehicles that cause them to result in a fractured foreground mask, notably their windows and wheels which look similar to road from the perspective of the background subtractor, have variable size depending on the vehicle's distance from the node. Hence, the structuring element used for augmenting the foreground mask should match the scale of the disconnection in entities which is determined by the region of interest specified by the natural performance of the background subtractor. 

To find the optimal morphological structuring element shape, size and number of iterations requires greater testing than selecting a median filter. These operations can become computationally expensive depending on the size of the structuring element and number of iterations so minimising these factors is important. The OpenCV function, \emph{exmorphology} provides three shapes of structuring element, ellipse, cross and rectangle and Figure \ref{fig:morph_testing} shows the comparative performance of these shapes on the same foreground mask for a 7x7 element size and varying iterations. Visual analysis of foreground entities consolidation reveals that the rectangular element for 3 iterations performs the best in the area of the mask where the best connection and separation of vehicle masks is already occurring. Figure \ref{fig:compare_closure} highlights one example of connecting blobs where other methods failed, in this case the blobs in the bounding boxes are closed successfully by the rectangular structuring element but not by the ellipse.

\begin{figure}[H]
    \centering
    \begin{subfigure}[b]{0.45\textwidth}
        \includegraphics[width=\textwidth]{design/detection/calibration/rect_3_edit}
        \caption{Closure with rectangle.}
    \end{subfigure}
    \begin{subfigure}[b]{0.45\textwidth}
        \includegraphics[width=\textwidth]{design/detection/calibration/ellipse_3_edit}
        \caption{Closure with ellipse.}
    \end{subfigure}
    \captionsetup{format = hang}
    \caption{Comparison of morphological closure using a 7x7 rectangular and elliptical structuring element for 3 iterations.}
    \label{fig:compare_closure}
\end{figure}


\begin{figure}[H]
    \begin{tabular}{
        >{\centering\arraybackslash}m{0.4cm}
        >{\centering\arraybackslash}m{4.5cm}
        >{\centering\arraybackslash}m{4.5cm}
        >{\centering\arraybackslash}m{4.5cm}}
          & Ellipse & Cross & Rectangle \\
        1 
        &
        \begin{subfigure}[b]{0.3\textwidth}
            \includegraphics[width=\textwidth]{design/detection/morphology/ellipse_1}
            % \captionsetup{format = hang}
        \end{subfigure} &
        \begin{subfigure}[b]{0.3\textwidth}
            \includegraphics[width=\textwidth]{design/detection/morphology/cross_1}
            % \captionsetup{format = hang}
        \end{subfigure} &
        \begin{subfigure}[b]{0.3\textwidth}
            \includegraphics[width=\textwidth]{design/detection/morphology/rect_1}
            % \captionsetup{format = hang}
        \end{subfigure} \\
        3 &
        \begin{subfigure}[b]{0.3\textwidth}
            \includegraphics[width=\textwidth]{design/detection/morphology/ellipse_3}
            % \captionsetup{format = hang}
        \end{subfigure} &
        \begin{subfigure}[b]{0.3\textwidth}
            \includegraphics[width=\textwidth]{design/detection/morphology/cross_3}
            % \captionsetup{format = hang}
        \end{subfigure} &
        \begin{subfigure}[b]{0.3\textwidth}
            \includegraphics[width=\textwidth]{design/detection/morphology/rect_3}
            % \captionsetup{format = hang}
        \end{subfigure} \\
        5 &
        \begin{subfigure}[b]{0.3\textwidth}
            \includegraphics[width=\textwidth]{design/detection/morphology/ellipse_5}
            % \captionsetup{format = hang}
        \end{subfigure} &
        \begin{subfigure}[b]{0.3\textwidth}
            \includegraphics[width=\textwidth]{design/detection/morphology/cross_5}
            % \captionsetup{format = hang}
        \end{subfigure} &
        \begin{subfigure}[b]{0.3\textwidth}
            \includegraphics[width=\textwidth]{design/detection/morphology/rect_5}
            % \captionsetup{format = hang}
        \end{subfigure} \\
    \end{tabular}
    \captionsetup{format = hang}
    \caption{Morphological closing using different 7x7 structuring elements and iterations.}
    \label{fig:morph_testing}
\end{figure}

\chapter{System Calibrations}

This section of the report details the different parameters for calibrating the computer vision algorithm and how they affect the algorithms behavior. Calibration must be performed for each unique traffic environment has different properties to ensure the correct function of the system. 

\section{Calibration Parameters}

The following is a list of calibration parameters and their description. In the system they are simply stored in a text file that the system reads upon execution.

\begin{itemize}
\item Gaussian Mixture Model:
    \begin{itemize}
        \item \textbf{training}: (\emph{Boolean}) Refers to whether or not the Gaussian Mixture Model for background subtraction should be trained on a number of frames first.
        \item \textbf{history}: (\emph{Integer}) Is the number of frames that the Gaussian Mixture Model for background subtraction considers when updating its clusters.
        \item\textbf{shadows}: (\emph{Boolean}) Specifies if the shadows cluster should be generated for the mixture model. Generally it should be as the tester should be seeking to account for shadows and removing them via thresholding.
    \end{itemize}

\item Testing: 
    \begin{itemize}
        \item \textbf{paid\_test}: (\emph{Boolean}) Used for testing on prerecorded video, runs through the frames at a high frame rate so that tester can observe system response quickly.
        \item \textbf{print\_positive}: (\emph{Boolean}) Specifies if the ID of centroids that have been counted in the positive direction should be printed to terminal.
        \item \textbf{print\_negative}: (\emph{Boolean}) Specifies if the ID of centroids that have been counted in the negative direction should be printed to terminal.
        \item \textbf{test\_length}: (\emph{Integer}) Specifies the number of frames that of a test video the system should be run for. This is useful for repeatedly observing the same section of a video to refine calibration.
        \item\textbf{frame\_wait}: (\emph{Integer}) Specifies the amount of time before processing the next frame for a test video. 
    \end{itemize}

\item Morphology and Filtering:
    \begin{itemize}
        \item\textbf{med\_size}: (\emph{Integer}) Specifies the size of the median filter kernel used to remove noise after background subtraction is performed. This value must be an odd number or else an error will be thrown as the function used requires that the filter be square with a center cell.   
        \item\textbf{morph\_shape}: (\emph{Integer}) Specifies the shapes of the structuring element in the morphology phase of the detection algorithm. 0 - Rectangular, 1 - Ellipse and 2 - Disc. 
        \item\textbf{morph\_size}: (\emph{Integer}) Specifies the size of the morphological element used in morphology phase of detection algorithm.
        \item\textbf{morph\_iter}: (\emph{Integer}) Specifies the number of iterations to be perform the morphology operations in the detection algorithm. Can increase the computational load significantly. 
    \end{itemize}

\item Object Tracking:
    \begin{itemize}
        \item \textbf{otherside\_centroids}: (\emph{Integer}) Specifies the number of frames that a centroids has to appear on the opposite side of a count line if it's to be counted as a vehicle crossing. This is to ensure that a centroid actually crosses the count line threshold.
        \item \textbf{history\_count}: (\emph{Integer}) Specifies the number of frames that a centroid has to exist for before it can be counted.
        \item\textbf{min\_width}:(\emph{Integer}) Specifies the minimum width in pixels that a blob can be considered a vehicle.
        \item\textbf{max\_width}:(\emph{Integer}) Specifies the maximum width in pixels that a blob can be considered a vehicle.
        \item\textbf{min\_height}:(\emph{Integer}) Specifies the minimum height in pixels that a blob can be considered a vehicle.
        \item\textbf{max\_height}:(\emph{Integer}) Specifies the maximum height in pixels that a blob can be considered a vehicle.
        \item\textbf{max\_dist}: (\emph{Integer}) Specifies the maximum distance in pixels that a new centroid can be located from an old one and be considered the old centroid's new position.
        \item\textbf{min\_dist}: (\emph{Integer}) Specifies the minimum distance in pixels between two centroids and them not be combined into one centroid. 
        \item\textbf{missing}: (\emph{Integer}) Specifies the number of frames that a centroid can be missing before that centroid is considered either left the frame or lost and removed from the current list of centroids.
        \item\textbf{num\_lines}: (\emph{Integer}) Specifies the number of count threshold lines in use. The minimum value is 1 and the maximum is 2. 
        \item\textbf{count\_line1\_p1}: Formatted as a point (x,y), specifies the start point of the first count line.
        \item\textbf{count\_line1\_p2}: Formatted as a point (x,y), specifies the end point of the first count line.
        \item\textbf{count\_line2\_p1}: Formatted as a point (x,y), specifies the start point of the second count line.
        \item\textbf{count\_line2\_p2}: Formatted as a point (x,y), specifies the end point of the second count line.
        \item\textbf{speed\_line1\_p1 }: Formatted as a point (x,y), specifies the start point of the first speed line.
        \item\textbf{speed\_line1\_p2 }: Formatted as a point (x,y), specifies the end point of the second speed line one.
        \item\textbf{speed\_line2\_p1 }: Formatted as a point (x,y), specifies the start point of the first speed line one.
        \item\textbf{speed\_line2\_p2 }: Formatted as a point (x,y), specifies the end point of the second speed line one.
    \end{itemize}

\item Database Storage:
    \begin{itemize}
        \item\textbf{storage\_interval}: (\emph{Integer}) Specifies how many frame should pass before the vehicle and speed count should be stored in the database.
        \item\textbf{multilane}: (\emph{Boolean}) Specifies whether or not there's multiple lanes of traffic for the purpose of how many database tables need to be updated as there's a table for each direction of traffic. 
    \end{itemize} 

\item Camera and Frame Setup:
    \begin{itemize}
        \item\textbf{traffic\_orientation}: (\emph{Boolean}) Specifies the direction of traffic relative to the camera setup orientation. True means traffic is moving up/down and False means traffic is moving left/right. 
        \item\textbf{scale\_percent}: (\emph{Integer}) Specifies the how much to scale the original input video. This will affect how quickly the algorithm processes each frame as it changes the number of pixels in a frame.
        \item\textbf{rotate}: (\emph{Integer}) Sometimes input video requires rotation so this parameter specifies that rotation to be either 0 or 180 degrees.
    \end{itemize} 

\item Graphics:
    \begin{itemize}
        \item\textbf{graphics}: (\emph{Boolean}) Sets whether or not to print any graphics onto the output video feed of the detection algorithm.
        \item \textbf{refresh\_count}: (\emph{Boolean}) Specify if the vehicle count values which are printed on the live video during processing should be reset to zero each time the vehicle volume count is stored in the database.
        \item \textbf{grid}: (\emph{Boolean}) Sets an onscreen grid to be printed on the live video feed. Useful while calibrating to help observe relative size of vehicles and blobs.
        \item \textbf{frame\_count}: (\emph{Boolean}) Sets if the current frame count of a test video should be displayed on the live feed. 
        \item \textbf{boxes}: (\emph{Boolean}) Sets if bounding boxes should be draw around the vehicles and blobs in video feed.
        \item\textbf{centroids}: (\emph{Boolean}) Sets if the object centroids should be drawn onto the video feed.
        \item\textbf{count\_line}: (\emph{Boolean}) Sets if the count line(s) should be drawn onto the video feed.
        \item\textbf{count\_graphics}: (\emph{Boolean}) Sets if the vehicle volume counts should be drawn onto the video feed.
        \item\textbf{pos\_pos}: Formatted as (x,y) is a point that specifies the location of the positive direction vehicle counts on the output video feed.
        \item\textbf{neg\_pos}: Formatted as (x,y) is a point that specifies the location of the negative direction vehicle counts on the output video feed.
        \item\textbf{singlelane}: (\emph{Boolean}) Specifies the direction of traffic flow if there's only one direction of flow. True means the traffic is travelling downward or to the right, False means the traffic is travelling upward or to the left.
    \end{itemize}
\end{itemize}
